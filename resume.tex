\chapter*{Résumé}                      % ne pas numéroter
\phantomsection\addcontentsline{toc}{chapter}{Résumé} % inclure dans TdM
\begin{otherlanguage*}{french}
  La curiethérapie est une modalité particulière de traitement en radiothérapie qui se différencie de la radiothérapie externe par l’implantation intratumorale, de façon permanente ou temporaire, de sources radioactives scellées. Cette modalité de traitement a bénéficié des avancées technologiques telles que l’utilisation des projecteurs de sources et le développement de nouveaux algorithmes d’optimisation de dose qui ont contribué à l’amélioration de la qualité des plans de traitement. Cependant, le processus de planification de traitement en curiethérapie, indépendamment de l’algorithme d’optimisation implanté dans le système de planification de traitement (TPS), requiert toujours une forte interaction entre le planificateur et le TPS. Cette forte interaction, non seulement, augmente la durée de la planification, mais conduit aussi souvent à des plans finaux dont la qualité est tributaire du jugement et de l’expérience du planificateur.  Le présent a consisté à développer des modèles pour l’optimisation de la qualité des plans en curiethérapie de la prostate haut débit de dose, optimisation basée sur les paramètres géométriques spécifiques à chaque patient, grâce au formalisme de l’analyse de frontière stochastique, une méthode empruntée de l’économie. Les paramètres géométriques qui ont permis de caractériser chaque plan sont : le volume des différentes structures (CTV, organes à risque), leurs proximités mutuelles par le biais de la distance de Hausdorff, et la divergence des cathéters dans la prostate. Les modèles ainsi développés constituent un outil d’aide à la planification en prédisant à l’avance (au début du processus de planification) le meilleur compromis qu’il est possible d’atteindre entre la couverture du volume tumorale et la limitation de la dose aux organes à risque alentour. Les modèles ont été développés pour les indices dosimétriques d’intérêt analysés en clinique pour la validation des plans pour chacune des structures telles que : la prostate, la vessie, le rectum et l’urètre. Les résultats obtenus sur la base d’un échantillon de 495 patients montrent que ces modèles peuvent aider le planificateur à personnaliser le processus d’optimisation sur la base du profil de paramètres géométriques de ce dernier, minimisant ainsi l’impact du jugement et l’expérience du planificateur sur la qualité du plan final. Leur utilisation peut être étendue comme un outil robuste pour la sélection de meilleurs plans pour les études de type \enquote{knowledge-based study}. Le projet doit cependant se poursuivre par l’exploration d’autres paramètres géométriques et par la validation clinique de ceux-ci avant leur implémentation en routine clinique.
\end{otherlanguage*}
