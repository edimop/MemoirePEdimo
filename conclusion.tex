\chapter{Conclusion}         % ne pas numéroter
%\phantomsection\addcontentsline{toc}{chapter}{Conclusion et développements futurs} % dans TdM
% * <larchamb@gmail.com> 2017-08-30T18:34:15.098Z:
% 
% > \chapter*{Conclusion et développements futurs}         % ne pas numéroter
% Voir commentaire de l'introduction: je préfère que ces sections soient numérotés
% 
% ^ <larchamb@gmail.com> 2017-08-30T18:34:49.476Z.
\lettrine[nindent=0em,lines=2]{L}{}a radiothérapie fait partie des différentes techniques principales de lutte contre le cancer, d’où sa parfaite intégration dans les différentes stratégies de traitement multidisciplinaires actuelles. Il s’agit d’une discipline qui est en constante évolution, bénéficient ainsi de nouveaux développements sur le plan de l’imagerie, de l’informatique, ainsi que des équipements de traitement. Particulièrement en curiethérapie, l’arrivée de l’imagerie échographique transrectale pour l’implantation des cathéters dans les volumes cibles fût une étape déterminante pour faire entrer la curiethérapie dans le $21^{\grave{e}me}$ siècle, suivie de l’arrivée des projecteurs de sources (curiethérapie HDD et débit de dose pulsé), et le développement de nouveaux algorithmes d’optimisation de dose. Ces mutations ont contribué à améliorer la prise en charge des patients dans l’atteinte des objectifs cliniques, mais ont aussi complexifié les pratiques, ce qui justifie l’évolution des méthodes contrôle de qualité permettant de garantir avec un niveau de confiance suffisant la cohérence d’une prescription médicale et l’exécution sécuritaire de celle-ci, afin de délivrer la dose prescrite au volume cible, tout en minimisant la dose aux tissus normaux alentour. Malgré ces évolutions remarquables, la qualité du plan qui résulte de la phase d’optimisation de la dose reste toujours tributaire de l’expérience et le jugement du planificateur. En effet, un plan pour être jugé cliniquement acceptable, doit respecter les recommandations sur la couverture du CTV (V$_{100}$ > 90\%) et les limites de dose aux OARs, à savoir, V$_{75}$ < 1 cm$^{3}$ (vessie, rectum) et (V$_{125}$ < 1 cm$^{3}$, D$_{10}$ < 118\% de la dose de prescription) pour l’urètre. On pourrait cependant se poser la question si un plan jugé clinique est forcément optimal. À partir de quel point un plan peut-il être jugé optimal dès lors qu’il est clinique? Le planificateur ne dispose d’aucun outil objectif lui permettant de répondre à cette question, il décide donc sur le caractère optimal d’un plan sur la base de son expérience et son jugement, d’où l’intérêt de ce projet; développer les modèles de frontière stochastique pour l’amélioration de la qualité des plans en curiethérapie haut débit de dose pour le cancer de la prostate. L’analyse de frontière stochastique qui est l’outil sur lequel repose le présent projet est un formalisme dont la paternité revient aux économistes. Lesdits modèles devront guider le processus de planification en prédisant, sur la base du profil de paramètres géométriques spécifiques à chaque patient, les paramètres dosimétriques qui caractérisent la couverture du CTV, ainsi que la minimisation de la dose aux OARs. Il sera donc possible grâce à ces modèles de minimiser l’impact du jugement et l’expérience propre à chaque planificateur sur la qualité du plan final. Quatre modèles ont été construits sur les paramètres dosimétriques qui présentent une variabilité inter-patient, sur la base d’un échantillon de 495 plans des patients traités en une fraction de 15 Gy sous images CT (complément de la radiothérapie externe).
Les résultats des modèles optimisés sur la base de cet échantillon montrent que ceux constituent un outil fiable pour la prédiction des paramètres dosimétriques pour le CTV, la vessie et l’urètre. Les modèles ont permis de montrer la variabilité de la qualité des plans finaux pour des patients ayant un profil de paramètres géométriques semblables. Cette variabilité est plus importante pour la couverture du CTV que pour la limitation de la dose aux OARs (vessie, urètre). Cette grande variabilité témoigne la signature du jugement et l’expérience inhérente à chaque planificateur, sur la qualité du plan final. Les présents modèles peuvent aider à minimiser cet effet, puisque ceux fournissent des informations sur la dosimétrie atteignable sur chaque patient; ce qui va motiver le planificateur à poursuivre le processus d’optimisation, afin d’obtenir ou d'approcher les indices dosimétriques prédits par les modèles, ou inversement, c’est-à-dire, arrêter le processus d’optimiser au bon moment (gestion efficace du temps). Le caractère subjectif sur le concept du meilleur compris peut également être limité grâce aux présents modèles, le meilleur compromis correspondant au cas pour lequel, chaque plan se trouverait ou serait proche des frontières des différents modèles. Le modèle optimisé pour le rectum ne contient aucune information prédictive puisque la composante de l’efficience technique est nulle. Ceci s’explique par le fait que les physiciens, sous recommandation des radio-oncologues s’évertuent beaucoup dans l’optimisation de cet organe, parfois au détriment de la vessie, ceci justifie également la faible variabilité inter-planificateur pour l’optimisation de cet organe, en comparaison avec la couverture de CTV. Outre l’information prédiction contenue dans les modèles, ceux-ci peuvent servir pour d’autres applications, par exemple, ils constituent un outil robuste dans la sélection de meilleurs plans dans un échantillon existant. \\
En conclusion, les résultats du présent projet constituent une avancée dans la démarche visant à développer des modèles plus complets et cliniquement fiables. En effet, les modèles développés doivent passer par une phase de test clinique. D’autre part, ils pourraient aussi être améliorés si les différents points soulevés dans la section (\ref{developementsfuturs}) du chapitre \ref{chapitre2} sont étudiés et intégrés dans le processus d’optimisation de ces derniers. En fin, les résultats présentés dans ce travail reposent sur un échantillon de patients sous images CT, la question quant à leurs validités sous images échographiques mérite une investigation. 