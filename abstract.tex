\chapter*{Abstract}                      % ne pas numéroter
\phantomsection\addcontentsline{toc}{chapter}{Abstract} % inclure dans TdM
\begin{otherlanguage*}{english}
Brachytherapy is a special modality of radiotherapy for cancerous tissues treatment. Unlike of external radiotherapy, this form of radiation therapy modality uses sealed radiation sources positioned permanently or temporarily within (or close to) the treatment volume. Brachytherapy treatment modality has benefited from technological advances such as the use of Remote-afterloading units and the development of new dose optimization algorithms that led to the improvement of treatment quality plans. However, the treatment planning process, regardless of the optimization algorithm used for dose calculation in the treatment planning system (TPS), still requires a strong interaction between the planner and the TPS. This strong interaction not only increases the planning time, but also often leads to final plans whose quality depends on the planner's judgment, as well as planner's experience. The goal of the current project was to develop models for the optimization of the quality of high dose rate prostate brachytherapy plans, based on patient’s specific geometric parameters, using stochastic frontier analysis, a method of economic modeling. Geometric parameters involved in the modeling process are the volume of contoured structures such as the clinical target volume (CTV) and organs at risk (OARs); the Hausdorff distance between CTV and OARs, and a third parameter measuring the degree of non-parallelism of catheters within the prostate. The built models are expected to be helpful in the treatment planning process by predicting dosimetric parameters values attainable at the starting point of the treatment planning. They will provide valuable indications in advance, on the level of dose reduction to OARs, as well as, the target volume coverage achievable. Models were built for dosimetric parameters of interest analyzed in the clinic for clinical validation of plans for each structure (prostate, bladder, rectum and urethra). The modelling results based on a dataset of 495 plans show that the developed models can be helpful to assist planner in the optimization process based on the geometric parameters profile of each plan, thus minimizing the impact of the judgment and the planner's experience on the final quality plan. Furthermore, their use can be extended as an accurate means for selecting optimized plans for a knowledge-based study. However, further research is required in order to investigate others geometric parameters, as well as, for the clinical benchmarking the performance of the developed models before their implementation in a clinical setting.
\end{otherlanguage*}
